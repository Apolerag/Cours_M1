\documentclass[a4paper,10pt]{article}
\usepackage[utf8x]{inputenc}
\usepackage{float}


%%%%%%%%%%% PACKAGES %%%%%%%%%%%%%%%%%%%

\usepackage[headinclude,footexclude]{typearea}

\usepackage{fullpage}
\usepackage[english,frenchb]{babel}
\usepackage[final]{graphicx}
\usepackage{scrpage2,lastpage,tikz,url}

\newcommand{\logos}{%
  \begin{tikzpicture}[remember picture,overlay]
    \node [yshift=-1cm] at (current page.north) [below] 
      {\includegraphics[height=15mm]{logo_lip}%\hspace*{1cm}
%\includegraphics[height=15mm]{logoverimag}

};
 \end{tikzpicture}}
\newcommand{\adresse}{%
  \begin{tikzpicture}[remember picture,overlay]
    \node [yshift=1cm] at (current page.south) [above] 
       {\begin{tabular}{c}

        \textsf{\textup{\color{blue}
          LIP -- UMR CNRS / ENS Lyon / UCB Lyon 1 / INRIA -- 69007
          Lyon
        }}\\
        \textsf{\textup{\color{blue}
         %- +33 (0)3 59 57 78 24
         % -- Fax. : +33 (0)3 28 77 85 37
          Contact E-mail : Laure.Gonnord@ens-lyon.fr
        }}
      \end{tabular}};
 \end{tikzpicture}}



%%%%%%%%%%%%%%%%%%% TITRE


\ohead[]{Analyses ``légères'' de pointeurs pour programmes C et
  tranformation de programmes}
\chead[]{}
\ihead[]{Laure Gonnord}
\ofoot[]{\thepage/\pageref{LastPage}}
\cfoot[]{}
\pagestyle{scrheadings}


\title{Proposition de Stage de Recherche:\\
Analyses ``légères'' de pointeurs pour programmes C}
\author{Laure Gonnord -- Maître de conférences Université Lyon1\\
%  co-encadrement Paul Feautrier -- Professeur Émérite Éns Lyon \\
% Membre de l'équipe Émeraude (CNRS/LIFL)\\
% Membre extérieur de l'équipe COMPSYS (INRIA/Éns Lyon/LIP)\\
%\url{Laure.Gonnord@ens-lyon.fr}
}
\date{}

%%%%MACROS ICI

\begin{document}
\maketitle
\logos \adresse

\section{Contexte}

L'avancée technologique des plateformes matérielles, avec des
caractéristiques toujours plus complexes, rend nécessaire des analyses
toujours plus fines du code source, que ce soit pour des buts
d'optimisation à la compilation (optimisation de taille mémoire, de
l'usage du cache\ldots) ou d'optimisations lors de l'exécution. 
\medskip

L'équipe Compsys a une expérience dans la transformation de programme
source à source, mais les programmes traités sont un noyau de
C. Nous désirons donc élargir le domaine d'applicabilité de nos
analyses, et en particulier travailler sur les représentations
alternatives des tableaux en C.

\medskip
En effet, on constate souvent que les pointeurs ne sont utilisés -en
C- que de façon rudimentaire~:
\begin{itemize}
\item transmission d'arguments par adresses
\item allocation dynamique de tableaux
\item descripteurs de tableaux
\item pseudo-indices.
\end{itemize}


\section{Objectif}
L'objectif est donc de réaliser un outil de détection de ces usages
simples et de transformer le programme initial en un programme
\emph{sémantiquement} équivalent qui sera ensuite utilisé comme entrée
de nos analyseurs, par exemple du genre de l'outil 
\texttt{c2fsm} \cite{FeautrierGonnord:tapas10}
qui transforme le programme en un automate interprété. 

\medskip Durant son stage de L3, C. Bacara a démarré une
implémentation à l'aide du framework
Clang/LLVM~\footnote{\url{llvm.org}}. Le résultat \footnote{Le rapport
  de stage se trouve à
  l'adresse~\url{http://laure.gonnord.org/pro/papers/rapportBacara.pdf}}
est un stockage dans une structure de données d'un certain nombre
d'informations concernant les pointeurs du programme (pointeurs
constants, pointeurs ``cachant'' des tableaux). Le stagiaire reprendra
cette implémentation, et après d'éventuelles améliorations,
s'attachera à coder la partie ``transformation de programme'', à
l'aide des outils disponibles dans le framework. À cette occation de
nombreux benchmarks seront réalisés.

\section{Prérequis}

Bonnes connaissances en C et en théorie de la compilation. Une
familiarité avec LLVM et son langage d'implémentation (C++) serait
appréciée.

% \section{Mots-clefs}
% Analyse statique de programmes C, interprétation abstraite, pointeurs,
% modèle mémoire, compilation source à source, LLVM.



\bibliographystyle{unsrt}
{\scriptsize \bibliography{bibStage}}


\end{document}